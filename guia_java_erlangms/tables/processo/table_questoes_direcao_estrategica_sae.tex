\begin{table}[!htb]
\centering
\small
\caption{Definiçao da direcao estrategica do SAE.}
\label{tab:tabela_direcao_estrategica_sae}
\begin{tabular}{|l|}
\hline
\begin{tabular}[c]{@{}l@{}}1) Quais os principais fatores de negócios que motivam a modernização?\\ Segundos os gestores do CPD/UnB, os principais fatores considerados para modernizar\\ o SAE foram: a) o custo de manutenção, já que este sistema legado é dividido em duas \\ partes. A primeira parte do sistema foi desenvolvida na linguagem VB e a segunda parte\\ em C\#. Algumas funcionalidades estão duplicadas entre as versões e quando ocorre uma\\ manutenção evolutiva, geralmente é necessário modificar os dois códigos fontes. \\ Atualmente, este sistema é mantido por dois analistas, sendo que cada analista mantem\\ uma versão do código fonte. b) O CPD deseja unificar os dois sistemas e ter apenas um único \\ sistema para manter. c) É necessário alinhar alguns novos requisitos com os stakeholders.\\ \\ 2) Quais os principais desafios tecnológicos para alcançá-los?\\ Não foi encontrado nenhum desafio tecnologico nesse projeto uma vez que os mantenedores\\ do sistema legado ainda estão no CPD.\\ \\ 3) Quais os objetivos a curto e médio prazo do projeto?\\ Foram identifiacdos três objetivos a curso prazo: a) remodelação tecnologica e a a migração \\ para a linguagem Java, visando obter uma única base de código para facilitar a manutenção. b)\\ revisar os requisitos do sistema junto aos usuários para que satisfaça as necessidades atuais. \\ c) testar a abordagem SOA proposta neste trabalho de dissertação em um projeto de modernização\\ real no CPD/UnB para verificar se a abordagem melhora a manutenibilidade dos sistemas.\\ Objetivos a médio prazo: Integrar o sistema SAE com o sistema SISRU, do restaurante universitário.\end{tabular} \\ \hline
\begin{tabular}[c]{@{}l@{}}4) Qual é a principal funcionalidade fornecida pelo sistema legado?\\ A principal funcionalidade é o preenchimento da avaliação sócio economica pelos estudantes \\ da UnB regularmente matriculados em disciplinas de cursos presenciais de graduação e pós-graduação.\\ \\ 5) Qual é a arquitetura em alto nível do sistema legado?\\ A versão VB do sistema possui uma arquitetura monolítica e as funcionnalidades estão implementadas\\ diretamente nos formulários.  A versão C\# possui uma arquitetura monolítica mas possui uma divisão três \\ camadas. A implementação da interface com o usuário está implementado em asp.net. Algumas regras\\ de negocios foram implementadas dentro do banco de dados através de stored procedures.\\ \\ 6) Quais interfaces com o usuário são utilizadas?\\ São utilizadas duas interfaces com o usuário. A versão VB do sistema é um sistema desktop enquanto que a\\ versão C\# é um sistema Web. O preenchimento do estudo socioeconomico é através da versão C\#. A versão\\ VB é utilizada pelos administradores do PNAES para fazer a gestão da avaliação e emissão de relatórios.\end{tabular}                                                                                                                                                                                                                                                                                                                                                                                                                                                                                                                                       \\ \hline
\begin{tabular}[c]{@{}l@{}}7) Quais são os usuários do sistema legado?\\ Administradores do programa PNAES e os estudantes da UnB.\end{tabular}                                                                                                                                                                                                                                                                                                                                                                                                                                                                                                                                                                                                                                                                                                                                                                                                                                                                                                                                                                                                                                                                                                                                                                                                                                                                                                                                                                                                                                                                                                                                                                        \\ \hline
\end{tabular}
\end{table}