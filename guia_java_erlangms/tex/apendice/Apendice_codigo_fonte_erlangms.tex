\section{Módulo msbus\_server\_worker}



\lstset{language=Erlang,
        basicstyle=\footnotesize,
        numbers=left,
        numberstyle=\footnotesize,
        tabsize=2,
        numbers=none,
        rulesepcolor=\color{blue}}
             
\begin{lstlisting} 

%%****************************************************************************
%% @title Module msbus_server_worker
%% @version 1.0.0
%% @doc Module responsible for processing \acrshort{HTTP} requests.
%% @author Everton de Vargas Agilar <evertonagilar@gmail.com>
%% @copyright ErlangMS Team
%%****************************************************************************


-module(msbus_server_worker).

-behavior(gen_server). 
-behaviour(poolboy_worker).

-include("../include/msbus_config.hrl").
-include("../include/msbus_schema.hrl").
-include("../include/msbus_http_messages.hrl").

%% Server API
-export([start/1, start_link/1, stop/0]).

%% gen_server callbacks
-export([init/1, 
		 handle_call/3, 
		 handle_cast/2, 
		 handle_info/1, 
		 handle_info/2, 
		 terminate/2, 
		 code_change/3]).

%% Client API
-export([cast/1]).

% State of server
-record(state, {
				worker_id = undefined,  	 %% identifier of worker
				lsocket   = undefined,		 %% socket of listener
				socket	  = undefined,		 %% socket of request
				allowed_address = undefined, %% Range of IPs that can access the server
				open_requests = []
			}).

-define(SERVER, ?MODULE).

%%====================================================================
%% Server API
%%====================================================================

start(Args) -> 
    gen_server:start_link(?MODULE, Args, []).
    
start_link(Args) ->
    gen_server:start_link(?MODULE, Args, []).
    
stop() ->
    gen_server:cast(?SERVER, shutdown).
 

%%====================================================================
%% Client API
%%====================================================================

%% @doc Send message for worker
cast(Msg) -> msbus_pool:cast(msbus_server_worker_pool, Msg).


%%====================================================================
%% gen_server callbacks
%%====================================================================

init({Worker_Id, LSocket, Allowed_Address}=Args) ->
    State = #state{worker_id = Worker_Id, 
				   lsocket = LSocket, 
				   allowed_address = Allowed_Address,
				   open_requests=[]},
    {ok, State, 0};

init(Args) ->
    %fprof:trace([start, {procs, [self()]}]),
    {ok, #state{}}.

handle_cast(shutdown, State) ->
    {stop, normal, State};

handle_cast({Socket, RequestBin}, State) ->
	NewState = trata_request(Socket, RequestBin, State),
	{noreply, NewState, 0};
	
handle_cast({HttpCode, Request, Result}, State) ->
	Worker = self(),
	Socket = Request#request.socket,
	inet:setopts(Socket,[{active,once}]),
	case is_integer(HttpCode) of
		true -> Code = HttpCode;
		false -> Code = 200
	end,
	case Code >= 400 of
		true -> Status = error;
		false -> Status = ok
	end,
	CodeBin = integer_to_binary(Code), 
	case Result of
		<<>> -> 
			Response = msbus_http_util:encode_response(<<"200">>, <<>>),
			envia_response(Code, ok, Request, Response);
		{ok, Content} -> 
			Response = msbus_http_util:encode_response(CodeBin, Content),
			case Status of
				error -> envia_response(Code, Content, Request, Response);
				_-> envia_response(Code, ok, Request, Response)
			end;
		{ok, Content, MimeType} -> 
			Response = msbus_http_util:encode_response(CodeBin, Content, MimeType),
			envia_response(Code, Status, Request, Response);
		{error, _Reason} -> 
			envia_response(Request, Result, State);
		{error, _Reason, _} -> 
			envia_response(Request, Result, State)
	end,
	{noreply, State#state{socket=undefined}}.

handle_call(_Msg, _From, State) ->
	{reply, _Msg, State}.

handle_info(timeout, State=#state{lsocket = undefined}) ->
	{noreply, State};

handle_info(timeout, State=#state{lsocket = LSocket, 
                                  allowed_address = Allowed_Address}) ->
	case gen_tcp:accept(LSocket, ?TCP_ACCEPT_CONNECT_TIMEOUT) of
		{ok, Socket} -> 
			% connection is established
			case inet:peername(Socket) of
				{ok, {Ip,_Port}} -> 
					case Ip of
						{127, 0, _,_} -> 
							{noreply, State#state{socket = Socket}};
						_ -> 
							case msbus_http_util:match_ip_address(Allowed_Address, Ip) of
								true -> 
									{noreply, State#state{socket = Socket}};
								false -> 
									gen_tcp:close(Socket),
									{noreply, State, 0}
							end
					end;
				_ -> 
					gen_tcp:close(Socket),
					{noreply, State, 0}
			end;
		{error, closed} -> 
			% ListenSocket is closed
			msbus_logger:info("Socket of listener is shutdown."),
			{noreply, State#state{lsocket = undefined}}; 
		{error, timeout} ->
			% no connection is established within the specified time
			%close_timeout_connections(State),
			{noreply, State#state{open_requests = []}, 0};
		{error, system_limit} ->
			msbus_util:sleep(3000),
			{noreply, State, 0};
		{error, PosixError} ->
			msbus_util:sleep(3000),
			{noreply, State, 0}
	end;

handle_info({tcp, Socket, RequestBin}, State) ->
	NewState = trata_request(Socket, RequestBin, State),
	{noreply, NewState, 0};

handle_info({tcp_closed, _Socket}, State) ->
	{noreply, State#state{socket = undefined}};

handle_info(_Msg, State) ->
   {noreply, State}.

handle_info(State) ->
   {noreply, State}.

terminate(_Reason, #state{socket = undefined}) ->
    ok;

terminate(Reason, #state{worker_id = Worker_id, socket = Socket}) ->
	gen_tcp:close(Socket),
    ok.

code_change(_OldVsn, State, _Extra) ->
    {ok, State}.

	
%%====================================================================
%% Internal functions
%%====================================================================

close_timeout_connections(#state{open_requests = Open_requests}) ->
	lists:foreach(fun(R) -> 
						case gen_tcp:controlling_process(R#request.socket, self()) of
							ok -> gen_tcp:close(R#request.socket);
							_ -> ok
						end
				  end, Open_requests).
	

trata_request(Socket, RequestBin, State) -> 
	Worker = msbus_pool:checkout(msbus_server_worker_pool),
	case msbus_http_util:encode_request(Socket, RequestBin, Worker) of
		 {ok, Request} -> 
			case gen_tcp:controlling_process(Socket, Worker) of
				ok -> 
					msbus_dispatcher:dispatch_request(Request),
					NewState = State#state{socket = undefined, 
										   open_requests = [Request | State#state.open_requests]};
				{error, closed} -> 
					NewState = State#state{socket=undefined};
				{error, not_owner} -> 
					NewState = State#state{socket=undefined};
				{error, PosixError} ->
					gen_tcp:close(Socket),
					NewState = State#state{socket=undefined}
			end;
		 {error, Request, Reason} -> 
			envia_response(Request, {error, Reason}, State),
			NewState = State#state{socket = undefined};
		 {error, invalid_http_header} -> 
			gen_tcp:close(Socket),
			NewState = State#state{socket = undefined}
	end,
	msbus_pool:checkin(msbus_server_worker_pool, Worker),
	NewState.

envia_response(Request, {ok, Result, MimeType}, _State) ->
	Response = msbus_http_util:encode_response(<<"200">>, Result, MimeType),
	envia_response(200, ok, Request, Response);

envia_response(Request, {error, notfound}, _State) ->
	Response = msbus_http_util:encode_response(<<"404">>, ?HTTP_ERROR_404),
	envia_response(404, notfound, Request, Response);

envia_response(Request, {error, no_authorization}, _State) ->
	Response = msbus_http_util:encode_response(<<"401">>, ?HTTP_ERROR_401),
	envia_response(401, no_authorization, Request, Response);

envia_response(Request, {error, invalid_payload}, _State) ->
	Response = msbus_http_util:encode_response(<<"415">>, ?HTTP_ERROR_415),
	envia_response(415, invalid_payload, Request, Response);

\end{lstlisting}


\section{Módulo msbus\_eventmgr}


\begin{lstlisting} 


%%********************************************************************
%% @title Module msbus_eventmgr
%% @version 1.0.0
%% @doc Module publisher/subscribe of ErlangMS.
%% @author Everton de Vargas Agilar <evertonagilar@gmail.com>
%% @copyright erlangMS Team
%%********************************************************************

-module(msbus_eventmgr).

-behavior(gen_server).

%% Server API
-export([start/0, stop/0]).

%% Client API
-export([adiciona_evento/1, 
		 registra_interesse/2, 
		 desregistra_interesse/2, 
		 lista_evento/0, 
		 lista_interesse/0, 
		 notifica_evento/2,
		 cancela_evento/1]).

%% gen_server callbacks
-export([init/1, 
	     handle_call/3, 
	     handle_cast/2, 
	     handle_info/2, 
	     terminate/2, 
	     code_change/3]).

-define(SERVER, ?MODULE).

-record(state, {lista_evento = [], 
			    lista_interesse = []}).


%%====================================================================
%% Server API
%%====================================================================

start() ->
    Result = gen_server:start_link({local, ?SERVER}, ?MODULE, [], []),
    Result.
 
stop() ->
    gen_server:cast(?SERVER, shutdown).
 
 
%%====================================================================
%% Client API
%%====================================================================
 
adiciona_evento(Evento) -> 
	gen_server:call(?SERVER, {adiciona_evento, Evento}). 
cancela_evento(Evento) -> 
	gen_server:call(?SERVER, {cancela_evento, Evento}). 
registra_interesse(Evento, Fun) -> 
	gen_server:call(?SERVER, {registra_interesse, Evento, Fun}). 
desregistra_interesse(Evento, Fun) -> 
	gen_server:call(?SERVER, {desregistra_interesse, Evento, Fun}). 
notifica_evento(QualEvento, Motivo) -> 
	gen_server:cast(?SERVER, {notifica_evento, QualEvento, Motivo}). 
lista_evento() -> 
	gen_server:call(?SERVER, msg_lista_evento). 
lista_interesse() -> 
	gen_server:call(?SERVER, lista_interesse). 


%%====================================================================
%% gen_server callbacks
%%====================================================================
 
init([]) ->
	%fprof:trace([start, {procs, [self()]}]),
    {ok, #state{}}.
    
handle_cast({notifica_evento, QualEvento, Motivo}, State) ->
	notifica_evento(State#state.lista_interesse, QualEvento, Motivo),
	{noreply, State};
    
handle_cast(shutdown, State) ->
    {stop, normal, State}.
    
handle_call({adiciona_evento, Evento}, _From, State) ->
	{Reply, NewState} = novo_evento(Evento, State),
	{reply, Reply, NewState};

handle_call({cancela_evento, Evento}, _From, State) ->
	{Reply, NewState} = cancela_evento(Evento, State),
	{reply, Reply, NewState};

handle_call({registra_interesse, Evento, Fun}, _From, State) ->
	{Reply, NewState} = novo_interesse(Evento, Fun, State),
	{reply, Reply, NewState};

handle_call({desregistra_interesse, Evento, Fun}, _From, State) ->
	{Reply, NewState} = remove_interesse(Evento, Fun, State),
	{reply, Reply, NewState};
    
handle_call(msg_lista_evento, _From, State) ->
	Reply = lista_evento(State),
	{reply, Reply, State};

handle_call(lista_interesse, _From, State) ->
	Reply = lista_interesse(State),
	{reply, Reply, State}.

handle_info(_Info, State) ->
    {noreply, State}.
 
terminate(_Reason, _State) ->
    ok.
 
code_change(_OldVsn, State, _Extra) ->
    {ok, State}.
    
    
%%====================================================================
%% Internal functions
%%====================================================================
    

existe_evento(Evento, State) ->
	lists:member(Evento, State#state.lista_evento).

existe_interesse(Interesse, State) ->
	lists:member(Interesse, State#state.lista_interesse).
    
novo_evento(Evento, State) ->
	case existe_evento(Evento, State) of
		true  -> {eeventojacadastrado, State};
		false -> {ok, State#state{lista_evento=[Evento|State#state.lista_evento]}}
	end.

cancela_evento(Evento, State) ->
	{ok, State#state{
		lista_evento = State#state.lista_evento -- [Evento], 
		lista_interesse = [I || {E,_} = I <- State#state.lista_interesse, E /= Evento]}}.

novo_interesse(Evento, Fun, State) ->
	case existe_evento(Evento, State) of
		true -> 
			case existe_interesse({Evento, Fun}, State) of
				true  -> {einteressejacadastrado, State};
				false -> {ok, State#state{
							lista_interesse=[{Evento, Fun}|State#state.lista_interesse]}}
			end;
		false -> {eeventonaocadastrado, State}
	end.

remove_interesse(Evento, Fun, State) ->
	case existe_interesse({Evento, Fun}, State) of
		true  -> {ok, State#state{
					lista_interesse = State#state.lista_interesse -- [{Evento, Fun}]}};
		false -> {einteressenaocadastrado, State}
	end.
		
notifica_evento([], _QualEvento, _Motivo) -> ok;
notifica_evento([{Evento, Fun} = _H|T], QualEvento, Motivo) ->
	case Evento == QualEvento of
		true ->
			try
				Fun(QualEvento, Motivo)
			after
				notifica_evento(T, QualEvento, Motivo)		
			end;
		false ->
			notifica_evento(T, QualEvento, Motivo)
	end.
	

\end{lstlisting}

\section{Módulo msbus\_dispatcher}

\begin{lstlisting} 


%%************************************************************************
%% @title Module dispatcher
%% @version 1.0.0
%% @doc Responsible for forwarding the requests to / from the \acrshort{REST} services.
%% @author Everton de Vargas Agilar <evertonagilar@gmail.com>
%% @copyright ErlangMS Team
%%************************************************************************

-module(msbus_dispatcher).

-behavior(gen_server). 
-behaviour(poolboy_worker).

-include("../include/msbus_config.hrl").
-include("../include/msbus_schema.hrl").
-include("../include/msbus_http_messages.hrl").

%% Server API
-export([start/0, start_link/1, stop/0]).

%% Client API
-export([dispatch_request/1]).

%% gen_server callbacks
-export([init/1, 
	     handle_call/3, 
	     handle_cast/2, 
	     handle_info/1, 
	     handle_info/2, 
	     terminate/2, 
	     code_change/3]).

% estado do servidor
-record(state, {}).

-define(SERVER, ?MODULE).

%%====================================================================
%% Server API
%%====================================================================

start() -> 
    gen_server:start_link({local, ?SERVER}, ?MODULE, [], []).
 
start_link(Args) ->
    gen_server:start_link(?MODULE, Args, []).

stop() ->
    gen_server:cast(?SERVER, shutdown).
 

%%====================================================================
%% Client API
%%====================================================================

dispatch_request(Request) -> 
	msbus_pool:cast(msbus_dispatcher_pool, {dispatch_request, Request}).

 
%%====================================================================
%% gen_server callbacks
%%====================================================================
 
init(_Args) ->
    createEtsControle(),
    %fprof:trace([start, {procs, [self()]}]),
    {ok, #state{}}.
 
createEtsControle() ->
    try
		ets:new(ctrl_node_dispatch, [set, named_table, public])
	catch
		_:_ -> ok
	end.
 
handle_cast(shutdown, State) ->
    {stop, normal, State};

handle_cast({dispatch_request, Request}, State) ->
	do_dispatch_request(Request),
	{noreply, State};

handle_cast(_Msg, State) ->
	{noreply, State}.
    
handle_call(Msg, _From, State) ->
	{reply, Msg, State}.

handle_info(State) ->
   {noreply, State}.

handle_info({Code, RID, Reply}, State) ->
	case msbus_request:get_request_em_andamento(RID) of
		{ok, Request} -> 
			msbus_request:registra_request(Request),
			msbus_eventmgr:notifica_evento(ok_request, {Code, Request, Reply}),
			{noreply, State};
		{erro, notfound} -> {noreply, State}
	end;

handle_info(_Msg, State) ->
   {noreply, State}.

terminate(_Reason, _State) ->
    ok.
 
code_change(_OldVsn, State, _Extra) ->
    {ok, State}.

	
%%====================================================================
%% Internal functions
%%====================================================================

%% @doc Dispatches the request to the service registered in the catalog
do_dispatch_request(Request) ->
	case msbus_catalogo:lookup(Request) of
		{Contract, ParamsMap, QuerystringMap} -> 
			case msbus_auth_user:autentica(Contract, Request) of
				{ok, User} ->
					case get_work_node(Contract#servico.host, 
									   Contract#servico.host,	
									   Contract#servico.host_name, 
									   Contract#servico.module_name, 1) of
						{ok, Node} ->
							Request2 = Request#request{user = User, 
													   node_exec = Node,
													   servico = Contract,
													   params_url = ParamsMap,
													   querystring_map = QuerystringMap},
							msbus_request:registra_request(Request2),
							msbus_eventmgr:notifica_evento(new_request, Request2),
							case executa_servico(Node, Request2) of
								ok -> ok;
								Error -> msbus_eventmgr:notifica_evento(erro_request, 
																		{servico, Request2, Error})
							end;
						Error -> 
							msbus_eventmgr:notifica_evento(erro_request, 
							                               {servico, Request, Error})
					end;
				{error, no_authorization} -> 
					msbus_eventmgr:notifica_evento(erro_request, {servico, 
					                                              Request, 
					                                              {error, no_authorization}})
			end;
		notfound -> 
			msbus_request:registra_request(Request),
			msbus_eventmgr:notifica_evento(erro_request, {servico, Request, 
			                                             {error, notfound}});
		Erro ->
			io:format("ERRO -> ~p\n", [Erro])
	end,
	ok.

%% @doc Runs the local service (Service written in Erlang)
executa_servico(_Node, Request=#request{servico=#servico{host='', 
														 host_name = HostName,	
														 module=Module, 
														 module_name = ModuleName, 
														 function_name = FunctionName, 
														 function=Function}}) ->
	try
		case whereis(Module) of
			undefined -> 
				Module:start(),
				apply(Module, Function, [Request, self()]);
			_Pid -> 
				apply(Module, Function, [Request, self()])
		end,
		msbus_logger:info("CAST ~s:~s em ~s {RID: ~p, URI: ~s}.", [ModuleName, 
																   FunctionName, 
																   HostName, 
																   Request#request.rid, 
																   Request#request.uri]),
		ok
	catch
		_Exception:ErroInterno ->  {error, servico_falhou, ErroInterno}
	end;

%% @doc Performs a service remotely
executa_servico(Node, Request=#request{servico=#servico{host = _HostList, 
														host_name = _HostNames,	
														module_name = ModuleName, 
														function_name = FunctionName, 
														module = Module}}) ->
	% Sends an asynchronous message to the service
	Msg = {{Request#request.rid, 
					   Request#request.uri, 
					   Request#request.type, 
					   Request#request.params_url, 
					   Request#request.querystring_map,
					   Request#request.payload,	
					   Request#request.content_type,	
					   ModuleName,
					   FunctionName}, 
					   self()
					  },
	{Module, Node} ! {{Request#request.rid, 
					   Request#request.uri, 
					   Request#request.type, 
					   Request#request.params_url, 
					   Request#request.querystring_map,
					   Request#request.payload,	
					   Request#request.content_type,	
					   ModuleName,
					   FunctionName}, 
					   self()
					  },
	msbus_logger:info("CAST ~s:~s em ~s {RID: ~p, URI: ~s}.", [ModuleName, 
															   FunctionName, 
															   atom_to_list(Node), 
															   Request#request.rid, 
															   Request#request.uri]),
	ok.

get_work_node('', _, _, _, _) -> {ok, node()};

get_work_node([], HostList, HostNames, ModuleName, 1) -> 
	get_work_node(HostList, HostList, HostNames, ModuleName, 2);

get_work_node([], _HostList, HostNames, _ModuleName, 2) -> 
	Motivo = lists:flatten(string:join(HostNames, ", ")),
	{error, servico_fora, Motivo};

get_work_node([_|T], HostList, HostNames, ModuleName, Tentativa) -> 
	case ets:lookup(ctrl_node_dispatch, ModuleName) of
		[] -> 
			Index = 1;
		[{_, Idx}] -> 
			% found a name that was used previously, we will use the next
			ets:delete(ctrl_node_dispatch, ModuleName),
			Index = Idx+1
	end,
	case Index > length(HostList) of
		true -> Index2 = 1;
		false -> Index2 = Index
	end,
	ets:insert(ctrl_node_dispatch, {ModuleName, Index2}),
	Node = lists:nth(Index2, HostList),
	case net_adm:ping(Node) of
		pong -> {ok, Node};
		pang -> get_work_node(T, HostList, HostNames, ModuleName, Tentativa)
	end.
		
%	
%is_node_alive(Node) -> 
%	case ets:lookup(ctrl_ping_cache, Node) of		
%		[] -> 		
%			Hit = net_adm:ping(Node);
%		[{Node, Time, Hit2}] -> 		
%			Time2 = calendar:datetime_to_gregorian_seconds(calendar:local_time()),			
%			case (Time2 - Time) < 2 of		
%				true -> io:format("hit\n\n"), Hit = Hit2;
%				false -> Hit = net_adm:ping(Node)
%			end		
%	end,		
%	NewTime = calendar:datetime_to_gregorian_seconds(calendar:local_time()),		
%	ets:insert(ctrl_ping_cache, {Node, NewTime, Hit}),
%	Hit.
	


\end{lstlisting}
