Este guia apresenta os conhecimentos necessários para trabalhar
com o serviço LDAP v3 no barramento de serviços ERLANGMS. O 
LDAP v3 refere-se a implementação protocolo
Lightweight Directory Access Protocol, ou LDAP,
um protocolo de aplicação aberto, livre de fornecedor e 
padrão de indústria que pode ser utilizado para oferecer
um "logon único" onde uma senha para um usuário é
compartilhada entre muitos serviços.

O barramento de serviços ERLANGMS foi desenvolvido 
pelo Analista em Tecnologia de Informação 
Everton de Vargas Agilar
no Mestrado Profissional em Computação Aplicada da UnB com 
o intuíto de facilitar a modernização e a integração de 
sistemas por meio de uma abordagem orientada a serviços. Entre 
os serviços disponíveis neste barramento, destaca-se
o serviço de autenticação de usuários com o uso do
protoloco LDAP v3.
