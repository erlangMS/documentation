Este guia sintetiza os conhecimentos que são necessários para trabalhar com a nova arquitetura 
Erlangms no ambiente Java. Este trabalho é resultado dos esforços realizados no Mestrado em Computação 
Aplicada pelo Analista Everton de Vargas Agilar do CPD/UnB.

A abordagem proposta por esta arquitetura
impõem o uso de um barramento de serviço desenvolvido na 
linguagem Erlang e um SDK (Software Development Kit) na linguagem 
que será utilizada para a implementação dos Web-Services, neste caso, o SDK ems-java 
para a linguagem Java. 

De forma muito resumida, a arquitetura ErlangMS tem o intuíto de facilitar a 
criação e a integração de sistemas através de uma abordagem
orientada a serviços no estilo arquitetural REST (Representational State Transfer). 
